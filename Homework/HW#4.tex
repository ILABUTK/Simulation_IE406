\documentclass{article}
\usepackage{multicol}
\usepackage{graphicx}
\usepackage[top=1.03in, bottom=0.51in, left=0.93in, right=0.92in]{geometry}

\usepackage{color}
%\usepackage[dvipsnames]{xcolor}

\usepackage{url}
\usepackage{hyperref}
\definecolor{linkcolour}{rgb}{0,0.2,0.6} 
\hypersetup{colorlinks,breaklinks,urlcolor=linkcolour, linkcolor=linkcolour}  
\usepackage{pdfpages}

%wrapfig
\usepackage{wrapfig}
\usepackage{lipsum}  % generates filler text

\renewcommand{\familydefault}{\sfdefault} 

\renewcommand{\labelenumi}{\Roman{enumi}. }

\begin{document}
\definecolor{orange}{rgb}{0.9648,0.4960,0}

\begin{figure}[ht]
\begin{minipage}[t]{0.40\linewidth}
\centering
\raisebox{-\height}{\includegraphics[width=2in]{UT-logo2.png}}

\label{fig:figure1}
\end{minipage}
\hspace{0.5cm}
\begin{minipage}[t]{0.5\linewidth}
\centering 
\vskip 0.2cm
\textcolor{orange}{\huge \bf HOMEWORK}
\vskip 0.2cm 
{\Large \bf IE 406: Simulation}
\vskip 0.2cm 
{\Large \bf }
\vskip 0.2cm 
{\Large \bf Dr. Xueping Li}

\end{minipage}
\end{figure}
{\bf
\begin{tabular}{ll}
%Course Section:	& IE 406 \\
\textcolor{orange}{------------------------------------------------------------------------------------------------------------------------------}
\end{tabular}
}

%%\vskip 0.3in

\begin{center}
{\textcolor{orange}{ \bf Homework Assignment (\#4)}}
\end{center}
\vskip 0.2in

%\begin{enumerate}

%\item 

\textcolor{orange}{\bf Goal::} \textit{{To assess the following: making plots; collecting statistics; using variables; using calendar; basic animation; multiple runs; define customized agents with attributes; etc.} }\\

\textcolor{orange}{\bf 1:}  Travelers arrive at the main entrance door of an airline terminal according to an exponential interarrival time distribution with mean 1.6 minutes, with the first arrival at time 0. The travel time from the entrance to the check-in is distributed uniformly between 2 and 3 minutes. At the check-in counter, travelers wait in a single line until one of five agents is available to serve them. There are two types of passengers, TSA pre-approved (10\%) and regular passengers (90\%). The check-in time (in minutes) for regular passengers follows a Weibull distribution with parameters beta=7.78 and alpha=3.91, while the check-in time for TSA pre-approved passengers follows a uniform distribution [1, 3] minutes. Upon completion of their check-in, they travel to their gates. Assume each passenger carries bags with weight between 15 to 45 pounds.
\\

\textit{\color{cyan}{Requirements}}: Create a simulation model, with animation (including the travel time from entrance to check-in), of this system. Run the simulation for 16 hours to determine the average time in system, number of passengers completing check-in, total weight of the bags from the passengers, and the average length of the check-in queue (use plots and texts).

\vskip 0.3in

%\item 
\textcolor{orange}{\bf 2:} An acute-care facility treats non-emergency patients (cuts, colds, etc.). Patients arrive according to an exponential interarrival time distribution with a mean of 11 (all times are in minutes). Upon arrival they check in at a registration desk staffed by a single nurse. Registration times follow a triangular distribution with parameters 6, 10, and 19. After completing registration, they wait for an available examination room; there are three identical rooms. Data shows that patients can be divided into two groups with regard to different examination times. The first group (55\% of patients) has service times that follow a triangular distribution with parameters 14, 22, 39. The second group (45\%) has triangular service times with parameters 24, 36, 59. Upon completion, patients are sent home. The facility is open 16 hours each day. 
\vskip 0.3in

\textit{\color{cyan}{Requirements}}:  Observe the average total time patients spend in the system (Length of Stay: LOS) using a plot. Animate the system. Conduct an experiment with 1000 runs. Compare with the LOS with the LOS obtained by one run (using seed number 1). Make a plot of the LOS for the 1000 runs.\\

\textcolor{orange}{\bf 3:} Study the ``Electronic Assembly/Test System (EATS)''. See Figure \ref{fig:eats}. \\

%\begin{wrapfigure}{l}{1.0\textwidth}
%\centering
%\includegraphics[width=0.55\textwidth]{EATS.png}
%\caption{\label{fig:eats} An Electronic Assembly/Test System illustration.}
%\end{wrapfigure}
\begin{figure} [htbp]
\centering
\includegraphics[width=0.66\textwidth]{EATS.png}
\caption{\label{fig:eats} An Electronic Assembly/Test System illustration.}
\end{figure}

The diagram is self-explanatory but DO NOT simply use the expressions (for the processing times) in the diagram in your model! You need to use the correct probability functions in AnyLogic. For the Weibull distribution, $\alpha=2.5$. For Part A, the average interarrival time is 5 minutes. For Part B, the average interarrival time is 30 minutes. 4 Part B arrive at the same time in a case. (You may change ?multiple agents per arrival? in the Source module). \\

\textit{\color{cyan}{Requirements}}: Build a base model to collect the data: 
\begin{itemize}
\item Average TIS for parts that are scrapped, salvaged and shipped
\item Average queue length of the ``rework'' station
\item Average utilization of the ``sealer'' machine
\end{itemize}

Also, add basic animation. Make 100 runs and show a plot for the average TIS for parts that are scrapped, salvaged and shipped. Display the grand average, standard deviation and half width. \\

\textcolor{cyan}{\bf How many runs are needed?} If the desired ``half width divided by the mean'' is less than 1\%? What confidence intervals will be? \\

\textcolor{orange}{\bf Submission Guideline:} Submit on CANVAS \& email! Summarize your work in a brief report, e.g., namely, summary.pdf, with screenshots (such as logic flow, charts, and animation) and explanation of the model (such that the TA may not even need to look into your source code). 

\end{document}

