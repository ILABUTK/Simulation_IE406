\documentclass{article}
\usepackage{multicol}
\usepackage{graphicx}
\usepackage[top=1.03in, bottom=0.51in, left=0.93in, right=0.92in]{geometry}

\usepackage{color}
\usepackage{url}
\usepackage{hyperref}
\definecolor{linkcolour}{rgb}{0,0.2,0.6} 
\hypersetup{colorlinks,breaklinks,urlcolor=linkcolour, linkcolor=linkcolour}  
\usepackage{pdfpages}

\renewcommand{\familydefault}{\sfdefault} 

\renewcommand{\labelenumi}{\Roman{enumi}. }

\begin{document}
\definecolor{orange}{rgb}{0.9648,0.4960,0}

\begin{figure}[ht]
\begin{minipage}[t]{0.40\linewidth}
\centering
\raisebox{-\height}{\includegraphics[width=2in]{UT-logo2.png}}

\label{fig:figure1}
\end{minipage}
\hspace{0.5cm}
\begin{minipage}[t]{0.5\linewidth}
\centering 
\vskip 0.2cm
\textcolor{orange}{\huge \bf HOMEWORK}
\vskip 0.2cm 
{\Large \bf IE 526: Simulation}
\vskip 0.2cm 
{\Large \bf Fall 2018}
\vskip 0.2cm 
{\Large  (Dr. Xueping Li)}

\end{minipage}
\end{figure}
{\bf
\begin{tabular}{ll}
%Course Section:	& IE 526 \\
\textcolor{orange}{------------------------------------------------------------------------------------------------------------------------------}
\end{tabular}
}

%%\vskip 0.3in

\begin{center}
{\textcolor{orange}{ \bf Homework Assignment (\#2)}}
\end{center}
\vskip 0.2in

%\begin{enumerate}
%
%\item 
%\textcolor{orange}{\bf 1:}  Simulation by-hand. Use the provided arrival time of the parts (arrival.txt) and the service times (service.txt) to simulate the operations
%in a manufacturing facility where \textbf{two} identical parallel machines are available. There is a single FIFO queue where parts wait to be processed. Simulate for 10 minutes. Make a table similar to the compact one in the handout \#40. 
%Summarize the Production, Average Waiting Time in Queue, Average Time in System, Time Average of Number of Parts in Queue, and Utilization. (Tip: Excel may be handy here). 
%\vskip 0.3in
%
%\textcolor{orange}{\bf Bonus:} Use a general purpose programming language (C, C++, Java, MatLab, etc.), a.k.a, your own simulation software, to simulate the operations and 
%calculate the Production, Average Waiting Time in Queue, Average Time in System, Time Average of Number of Parts in Queue, and Utilization. 
%
%\vskip 0.3in
%
%\item 
\textcolor{orange}{\bf 1:} Generate 10000 random numbers using the LCM (LCG) method. Feel free to use any programming language or software package. (Tip: R (\emph{Recommened}), Python or Excel may be handy here).
\begin{itemize}
\item Use $X_0$=27, a=17, c=9, and m=10000. 
\item  Plot a histogram of the above random numbers (RN) and comment on the quality of the RNs. 
\item  Use K-S test to test the uniformity of the first 10 RNs. Use Alpha=0.05. 
\item Generate 10000 variates following the exponential distribution with $\lambda$=10 based on the above RNs using inverse-transform. Plot a histogram of these variates.
\item  Use T-test to test the first 20 RNs. $H_0$: $\bar{X}$ = 0.5; $H_1$: $\bar{X} \neq$0.5; $\alpha$=0.01. 
\item {[\texttt{Optional. Bonus.}]} Use $X_0$=0, $a$=6364136223846793005, $c$=1442695040888963407, and m=10000. Generate 10000 random numbers and plot a histogram. 
\end{itemize}

\vskip 0.1in 

\textcolor{orange}{\bf 2:} Estimate the value of $\pi$ using random numbers only.  You may use the numbers that you generated in Question \#1, or you may generate more numbers for higher precision. (\textit{Tip:} you can conduct a Monte Carlo experiment by throwing darts at a board blindfolded. Or find other creative ways.)

\vskip 0.3in 

\textcolor{orange}{\bf Submission Guideline:} Summarize your work in a  report (in Word or PDF format), including charts, output, explanations and source code. Attach your source code. \\
1) Submit to Canvas. \\
2) Email your homework to \url{ie526.utk@gmail.com}. \\
When you have multiple files, please zip them
into one single file. 7zip (\url{http://www.7-zip.org/}) is recommended and .7z file extension is suggested. Note that occasionally, UTK email system may filter emails with attachments, especially when attachments are like .zip and/or with executable files. You should receive an email notification if your email goes through. 
\vskip 0.2in
%Also, please submit your homework to a shared Google Drive at \url{https://drive.google.com/drive/folders/0B92X26p0Cma-YmxjWFNqdk9HNnc?usp=sharing}. Note that you must use your UTK email/password to login into Google Drive to access this folder.  You will see something like``HW\#2'' under the folder. Create a folder like ``Netid\_Your Name'' and upload your unzipped files to that folder. You feel free to change the permission such that no one but the TA and Dr. Li can access that folder.



% 
%This course focuses on the construction of simulation models of real or
%  conceptual systems using the simulation software package \textbf{AnyLogic} 
%  while covering the analytical aspects of utilization of simulation models
%   and optimization techniques. Both \textbf{discrete event models} and \textbf{agent-based 
%   models} will be covered. \textbf{System dynamics models} will be briefly introduced.
%   
%\item \textcolor{orange}{\bf VALUE PROPOSITION:} Simulation is a powerful tool
%to analyze complex, dynamic and stochastic systems.  Ability to design, analyze, and
%interpret simulation models for such systems is crucial in making efficient business and
%  engineering decisions. This course will address the topics that are
%  essential for developing such skills. 
%
%\item \textcolor{orange}{\bf OBJECTIVES:} On completion of this course
% students will be able to correctly \textbf{design, analyze and interpret} the results
%  of computer simulation experiments using simulation software package. Specifically, they will: 
%  \begin{itemize}
%\item In general, be able to build simulation models using a computer software package;
%\item Understand the assumptions, strengths and weaknesses of simulation models;
%\item Analyze and translate problems into a form suitable for applying simulation strategies; and 
%\item Validate simulation models and optimize the system.
%\end{itemize}
%
%
%\item	\textcolor{orange}{\bf LEARNING ENVIRONMENT:} 
%
%%% [This section provides students with the instructor’s vision of what take place in the class.  It includes information about the methods of instruction, the role of the student, the role of the faculty member, what will take place in class, out of class, the tools for learning, etc. This section provides an opportunity to present the notion of shared responsibility for learning.  A useful table with examples is provided below: 
%
%\begin{tabular}{l|l}
%\hline
%Students Responsibility & Instructors Responsibility \\
%\hline
%\hline
% Be prepared for all classes & Be prepared for all classes \\
%\hline 
%Be respectful of others & Be respectful of the students \\
%\hline
%Actively contribute to the learning activities in class & Create and facilitate meaningful learning activities \\
%\hline
%Abide by the UT Honor Code & Evaluate all fairly and equally \\
%\hline
%\end{tabular}
%
%
%\item \textcolor{orange}{\bf TEXTBOOK (OPTIONAL) \& REFERENCES:} 
%
%{\bf AnyLogic 7 in Three Days}. Ilya Grigoryev. ISBN-10: 150893374X.\\
%{\bf The Big Book of Simulation Modeling}. Andrei Borshchev. ISBN-13: 978-0989573177.\\
%AnyLogic web site \url{http://www.anylogic.com}{}\\
%Winter Simulation Conference \url{http://www.wintersim.org}
%Institute of Industrial and Systems Engineers (IISE) Annual Conference \url{http://iienet2.org}
%
%{\bf Software:} AnyLogic v7.3.1/v.7.3.5. \\
%RESERVED MATERIALS will be made available at the library or course companion web sites as our course develops.
%
%\item \textcolor{orange}{\bf TECHNOLOGICAL RESOURCES:} 
%
% %% [This section includes information about course Blackboard Site, and any type of research / reference materials or technology (i.e., Online@UT, LiveText, or a bibliography) the student will need to use for the classroom.]
%
%All homework assignments and lecture slides will be posted on the course Blackboard Site or shared Google Drive folder. 
%
%\item \textcolor{orange}{\bf COURSE REQUIREMENTS, ASSESSMENT AND EVALUATION METHODS:} 
%
%\begin{enumerate}
%
%
%\item Students are responsible for announcements and material covered
%  in class.
%
%
%\item {\bf Homework and Quizzes.} Homework problems will be assigned
%   based on the covered material. Quizzes will be pop quizzes. 
%Assignments are due at the beginning of the class. No late work is accepted. Assignments are individual efforts based unless advised otherwise.  
%  
%
%\item {\bf Exams.} No written exams! Instead, you will be working on a team-based term project (rubrics will be given).
% Mini projects may be given as a take-home exam or a quiz. 
%
%
%\item {\bf Attendance} is not mandatory. However, it is {\bf extremely important} to succeed in this course.
%
%\item {\bf Extra credit. } Each student may receive up to 3 extra credit points towards their final grade during the lecture activities
%  (i.e., class engagement, pop-quiz, grade lottery).  {\bf No extra credit} will be given for
%  any additional work or re-doing assignments.
%
%\item {\bf Grading } will be based on homework assignments (30\%), quizzes (30\%), and term projects (40\%). 
%
%%% To calculate your grade for the class, the following formula
%%%   will be used:
%
%%% $$
%%% G = 0.25 \cdot \frac{G_{Q}}{7} + 0.25\cdot G_{exam1} + 0.25 \cdot G_{exam2} + 0.25 \cdot G_{exam2} 
%%% $$
%
%%%   where $G_{Q}$  is the sum of all points received for the quizzes (0 to 700),
%
%%%  $G_{exam1}$ is the first exam grade, $G_{exam2}$ and $G_{exam3}$ are the second and third exam grades respectively. 
%
%%% \vskip 0.3in
%%%   For example, if you got 550 points out of 700 possible for your
%%%   homework assignments, 87 points on the first exam, 55 points on the
%%%   second exam, and 95 points on the final exam, then you final grade
%%%   will be:
%
%%%   \begin{equation*}
%%%     G = 0.25 * 550/7 + 0.25 * 87 + 0.25 * 55 + 0.25 * 95 = 78.89
%%% \end{equation*}
%
%\item {\bf Grading scale}: A (100-90), B+ (90-86), B(85-80), C+ (80-75), C (75-70), D (70-60), F (60-0)
%%\begin{tabular}{|c|c|c|c|c|c|}
%%\hline
%%& &A&100-91&A-&91-89\\
%%\hline
%%B+& 89-86 &B& 86-82 &B--& 82-80\\
%%\hline
%%C+& 80-77 &C& 77-72 &C--& 72-70\\
%%\hline
%%D+&70-67 &D& 67-62 &D--& 52-60\\
%%\hline 
%%F & 59-0 & & & & \\
%%\hline
%%
%%\end{tabular}
% 
%%\item Cell phones, loud music and noisy behavior  are not allowed.
%\item {\bf Policy:} When taking a quiz or homework assignment, cheating in any form will result in a grade of 0 for that exam. Plagiarism is defined as ``to steal and pass off (the ideas or words of another) as one's own, to use (another's production) without crediting the source'' and ``to commit literary theft: present as new and original an idea or product derived from an existing source'' as on Merriam-Webster. Online \url{http://www.m-w.com}. 
%
%\end{enumerate}
%
%%% [This section includes class attendance and tardiness policy, evaluation methods and grading system, i.e., rubrics, percentages, tests, quizzes, weighting, curve or UT grading distribution, grade appeals to instructor, etc.; it may also include the policy for incompletes and withdrawals.]  
%
%%% Exams and quizzes (how many, what kind, dates, final exam period, missed exams/makeup exams, etc.);
%%% Assignments/problem sets/projects/reports/research papers (general info, assessment criteria, format, policy for late or missed assignments);
%%% Other assignments (e.g., posting comments to discussion board) and extra credit opportunities).
%
%\item \textcolor{orange}{\bf HOW TO BE SUCCESSFUL IN THIS CLASS:} 
%  To maximize your success in this class:
%\begin{itemize}
%\item Invest some time and do your own work
%\item Attend every class 
%%\item Take notes and go over them before the next class
%\item Practice, practice, practice! 
%\item Attend office hours for extra help
%\item Ask questions during and after the class
%\item Discuss with your classmates / teammates
%\end{itemize}
%
%
%%% [This could span from general guidelines (e.g., estimated amount of time to spend on preparation, assignments, participation in class discussion, use of supplemental teaching materials, etc.) to more sophisticated rubrics related to student performance on various evaluation measures (e.g., exams, assignments, projects.]
%
%%% \item \textcolor{orange}{\bf COURSE FEEDBACK:} [This section may include methods of feedback to faculty member, such as formative feedback mechanisms during the semester.]
%
%%% \item \textcolor{orange}{\bf UNIVERSITY POLICIES:} 
%
%%% {\bf Please see the Campus Syllabus} (last page) for important information about the University Policies common across all courses at UT.
%
%%% {\bf Freedom to
%%%   Learn.}  The responsibility to secure and to respect general
%%%   conditions conducive to the freedom to learn is shared by all
%%%   members of the academic community. The university welcomes and
%%%   honors people of all races, creeds, cultures, and sexual
%%%   orientations, and values intellectual curiosity, pursuit of
%%%   knowledge, and academic freedom and integrity.
%
%
%%% {\bf Final Exam Policy.} Final exams must be given during the final exam period at
%%% the scheduled time, although alternative uses of the scheduled exam period may
%%% be designated by the instructor.
%%% \begin{itemize}
%%% \item Students are not required to take more than two written exams on any
%%% day. The instructor(s) of the last non-departmental exam(s) on that day
%%% must reschedule the student's exam during the exam period. It is the
%%% obligation of students with such conflicts to make appropriate
%%% arrangements with the instructor at least two weeks prior to the end of
%%% classes.
%%% \item All final exams must be given during the final exam period at the
%%% scheduled time, although alternative uses of the scheduled exam period
%%% may be designated by the instructor.
%%% \item No in-class written quizzes or tests counting more than 10 percent of
%%% the semester grade may be given the last five calendar days prior to the
%%% study period.
%%% \end{itemize}
%
%
%
%
%%% [This required section includes information about discrimination, scholastic dishonesty, cheating, and plagiarism policies (e.g., honor statement, consequences, examples, etc.). The honor statement is included on the Campus Syllabus available on the Provost and TennTLC websites, and the online UT catalog.]
%
%%%\item \textcolor{orange}{\bf STUDENTS WITH DISABILITIES POLICY:} Please see the Campus Syllabus below.
%
%%% \item \textcolor{orange}{\bf DEPARTMENT OR PROGRAM MISSION STATEMENT:} [OPTIONAL SECTION]
%
%\item   \textcolor{orange}{\bf  THE INSTRUCTOR RESERVES THE RIGHT TO REVISE, ALTER AND/OR AMEND THIS SYLLABUS, AS NECESSARY.  STUDENTS WILL BE NOTIFIED IN WRITING AND/OR BY EMAIL OF ANY SUCH REVISIONS, ALTERATIONS AND/OR AMENDMENTS.}
%\newpage
%\item \textcolor{orange}{\bf COURSE OUTLINE/ASSIGNMENT/UNITS OF
%  INSTRUCTION OR CLINIC SCHEDULE:}
%
%The following is a very tentative schedule for the course - we will
%probably change things as we proceed depending upon the progress we
%are able to make and the feedback from the students.
%\vskip 0.25in
%{
%\centering
%\begin{tabular}[c]{llll|}
%\hline
%\textcolor{blue}{} & \textcolor{blue}{Lecture Topics} & \textcolor{blue}{Time}\\
%\hline
%- & Introduction to simulation \& demos  & 1/2wk\\ 
%- & Fundamental simulation concepts (HW\#1)  & 1/2wk\\
%- & Simulation concepts \& statistics review & 1wk \\
%- & Random number generation (Quiz\#1, HW\#2) & 1/2wk \\
%- & A guided tour to AnyLogic (M/M/1 model: TIS using TimeMeasure module, Histogram, View etc.) & 1-2 wks \\
%- & AWTQ, WIP, Plots, collect statistics, animation, etc. HW\#3 , Quiz\#2& \\
%- & Modeling basic operations and inputs (Enterprise process models, Case Study PP/SP) & 2-3wks\\
%- & Modeling detailed operations (Schedules, Network, Variables, 3D animation, etc.)  HW\#4 & 2-3 wks\\
%- & Statistical output analysis & 1wk\\
%- & OptQuest for optimization and replications. Mini-project\#1 & 1wk\\
%- & Advanced topics (agent-based and system dynamics models) and case studies. Take-home quiz\#1. & 1-2 wks\\
%- & Term project presentation and demonstrations & 1-2wks\\
%\hline
%\end{tabular}
%}

  %% [This section typically includes a
  %% table with the tentative calendar, topics, and assignments, dates
  %% for exams and due dates, special events, etc.]


%% \item [EACH INSTRUCTOR MAY INSERT MISCELLANEOUS COURSE ELEMENTS HERE,
%%   AS DESIRED – NUMBERED IN SEQUENCE]: [OPTIONAL SECTION(S)]

%% \item \textcolor{orange}{\bf IMPORTANT DATES IN THE ACADEMIC CALENDAR
%%   FALL 2013:} 

%% \begin{tabular}{l|l}
%% Last Day to Add Classes & \\
%% Martin Luther King Day & \\
%% Labor Day (no classes) & \\ 
%% Last Day to Drop Course without WD & \\
%% Last Day to Drop Course without ``F'' & October \\
%% Spring Break/Fall Break (no classes) & October \\
%% Last Day of Classes & \\
%% Final Exams & \\
%% \end{tabular}



%% \item \textcolor{orange}{\bf  COURSE PLAN/UNIT PLANS:} [OPTIONAL SECTION/LANGUAGE.]
%\end{enumerate}

%%%\includepdf{CAMPUS-SYLLABUS.pdf}

\end{document}

