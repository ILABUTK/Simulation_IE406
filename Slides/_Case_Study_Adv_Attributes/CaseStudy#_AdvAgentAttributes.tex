\documentclass{article}
\usepackage{multicol}
\usepackage{graphicx}
\usepackage[top=1.03in, bottom=0.51in, left=0.93in, right=0.92in]{geometry}

\usepackage{color}
\usepackage{url}
\usepackage{hyperref}
\definecolor{linkcolour}{rgb}{0,0.2,0.6} 
\hypersetup{colorlinks,breaklinks,urlcolor=linkcolour, linkcolor=linkcolour}  
\usepackage{pdfpages}

%wrapfig
\usepackage{wrapfig}
\usepackage{lipsum}  % generates filler text

\renewcommand{\familydefault}{\sfdefault} 

\renewcommand{\labelenumi}{\Roman{enumi}. }

\begin{document}
\definecolor{orange}{rgb}{0.9648,0.4960,0}

\begin{figure}[ht]
\begin{minipage}[t]{0.40\linewidth}
\centering
\raisebox{-\height}{\includegraphics[width=2in]{UT-logo2.png}}

\label{fig:figure1}
\end{minipage}
\hspace{0.5cm}
\begin{minipage}[t]{0.5\linewidth}
\centering 
\vskip 0.2cm
\textcolor{orange}{\huge \bf CASE STUDY}
\vskip 0.2cm 
{\Large \bf Modeling and Simulation}
\vskip 0.2cm 
{\Large \bf Dr. Xueping Li }

\end{minipage}
\end{figure}
{\bf
\begin{tabular}{ll}
%Course Section:	& IE 526 \\
\textcolor{orange}{------------------------------------------------------------------------------------------------------------------------------}
\end{tabular}
}

%%\vskip 0.3in

\begin{center}
{\textcolor{orange}{ \bf Case Study: Advanced Use of Agent Types \& Parameters (Attributes)}}
\end{center}
\vskip 0.2in

%\begin{enumerate}

%\item 

\textcolor{orange}{\bf Goal::} \textit{To learn the following methods/techniques:}\\
\begin{itemize}

\item Define Agent (Entity) Type and attributes. Collect TIS through this method. Collect ``exact'' statistics for Time Average of WIP and Average TIS. 
\item Basic debugging techniques. \texttt{traceln()}
\item Define functions
\item Export/import data files via \texttt{file} module
\item ResourcePool module
\end{itemize}


\vskip 0.3in

\textcolor{orange}{\bf Problem Statement:}  Two types of customers, regular and VIP, arrive to a service station, following exponential interarrival times with mean 1 and 5 minutes, respectively.  There is a single queue, in which the VIP customers will be served first. (Note that this is not FIFO, but ``priority''-based.) The actual service station has two parallel desks. The service time for regular \& VIP customers are \texttt{uniform(2,4)} and \texttt{uniform(1,2)}, respectively.  \\

\vskip 0.2in
Find out the average cycle time for the two types of customers. 

\vskip 0.2in
Build a model and run 160 hours (assuming 7/24/365).

\vskip 0.2in \textcolor{orange}{Quick tips:}
\begin{itemize}
\item Define a single ``Customer'' agent type. Add parameters ``type'' and ``service\_time''. Use \texttt{agent.service\_time} to differentiate them. 
\item Assign ``agent.type = 1'' and ``agent.type = 5'' for regular/VIP customers, respectively. At the ``queue'', choose ``Advanced'' / ``Priority-based''. 
\item The above method is preferred! The method below is for illustration purpose.
\item Define ``RegularCustomers'' and ``VIPCustomers'' agent types. 
\item Define a function ``fGetPriority(a Agent)'' to find out the type. Such as: \\
\tt{int p=0; 

//if regular \\
if (a instanceof RegularCustomers) \\
   p =  (int)(((RegularCustomers) a).pPriority); \\
//if VIP \\   
if (a instanceof VIPCustomers)   \\
	p =  (int)(((VIPCustomers) a).pPriority);\\

return p;}
\end{itemize}
\begin{itemize}
\item Use \texttt{file.println()} method.
\item Use \texttt{traceln(time() + "," + agent.something} to print out some attributes. 
\end{itemize}

\vskip 0.3in
\textcolor{orange}{\bf Handout Companion:} Screencasts will be provided, along with ``in-class models''.  (``AdvDiffTypes'' \& ``Agent\_Attributes'').


\end{document}

